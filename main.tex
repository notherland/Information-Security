\documentclass[a4paper]{article}
\usepackage[utf8]{inputenc}
\usepackage[russian]{babel}
\usepackage[T2]{fontenc}
\usepackage[warn]{mathtext}
\usepackage{graphicx}
\usepackage{floatflt}
\usepackage[left=30mm, top=20mm, right=20mm, bottom=20mm, footskip=10mm]{geometry}


\graphicspath{ {images/} }
\usepackage{multicol}
\setlength{\columnsep}{2cm}


\begin{document}

\begin{titlepage}
	\centering
	\vspace{5cm}
	{\scshape\LARGE Московский физико-технический институт \par}
	\vspace{4cm}
	{\huge\bfseries Улучшения алгоритма Эль-Гамаля \par}
	\vspace{1cm}

	{\large выполнила студентка Б01-907 группы ФРКТ}\par
	\vspace{0.3cm}
	{\LARGE Широкова Ксения\par}\par
	\vspace{1cm}
\textit{\textbf{Аннотация : } С широким применением алгоритма Эль-Гамаля его безопасность обычно подвергается сомнению, и проблема становится все более серьезной. В данной статье будут рассмотрены исследования связанные с повышением устойчивости алгоритма Эль-Гамаля к выбранным атакам. А именно: улучшение для защиты от атаки подобранного шифротекста, устранение уязвимости схемы подписи Эль-Гамаля, в которой используется только одно случайное число и схема подписи Эль-Гамаля на конических кривых.
}
	\vfill

% Bottom of the page
	Долгопрудный, 2022 г.
\end{titlepage}

\section*{Базовый алгоритм}
\text{Для начала рассмотрим схему базового алгоритма Эль-Гамаля для работы в режиме шифрования и дешифрования и в режиме подписи и проверки подписи.}
\subsection*{Алгоритм генерации ключей}

\begin{itemize}
    \item Выбирается случайное простое число p.
    \item Выбирается целое число g такое, что ${g^{\phi\ (p)}\equiv 1{\pmod {p}}}$.
    \item Вычисляется ${ y=g^{x}{\pmod {p}}}$.
    \item Открытым ключом является (y, g, p), закрытым ключом является число x.
\end{itemize}

\subsection*{Работа в режиме шифрования}
M - сообщение, M меньше числа p \\
\\
\caption{Шифрование :}
\begin{itemize}
    \item Выбирается случайное целое число k - сессионный ключ, взаимно простое с (p - 1), такое, что 1 < k < p - 1.
    \item Вычисляются числа ${a = g^{k}{\pmod {p}}}$ и ${b = y^{k}M{\pmod {p}}}$.
    \item Шифротекстом является пара чисел (a, b).
\end{itemize}
\\
\caption{Расшифрование :}
\begin{itemize}
    \item Зная закрытый ключ x, получатель может вычислить исходное сообщение из шифротекста (a, b) по формуле:
            ${M = b * (a^{x})^{-1}{\pmod {p}}.}$\\
    \item При этом можно проверить, что: ${(a^{x})^{-1} = g^{-kx}{\pmod {p}}}$\\
    Тогда верно, что \\
    ${b * (a^{x})^{-1} = (y^{k} * M) * g^{-xk}\equiv (g^{xk} * M)g^{-xk}\equiv M{\pmod {p}}}.$
\end{itemize}

\subsection*{Работа в режиме цифровой подписи}
\text{При работе в режиме цифровой подписи предполагается, что используется хэш-функция h(), значения которой лежат в интервале от 1 до p - 1.} \\
\\
\caption{Подпись:}
\begin{itemize}
    \item Вычисляется дайджест сообщения M: m = h(M).
    \item Выбирается случайное число {1 < k < p - 1} взаимно простое с p - 1 и вычисляется ${r=g^{k}{\pmod {p}}}$
    \item Вычисляется число ${s\,=\,(m - x * r)k^{-1}{\pmod {p - 1}}}$, где ${k^{-1}}$ это обратное k по модулю p - 1.
    \item Пара чисел (r, s) является подписью.
\end{itemize}
\\
\caption{Проверка:}
\begin{itemize}
    \item Если не выполняется 0 < r < p и 0 < s < p - 1, то подпись недействительна
    \item Подпись считается верной, если выполняется сравнение: ${y^{r}r^{s} \equiv g^{m}{\pmod {p}}.}$
\end{itemize}

\subsection*{Корректность проверки}

\begin{itemize}
    \item Из определения s имеем ${m\,\equiv \,xr+sk{\pmod {p-1}}.}$
    \item Из малой теоремы Ферма: ${{\begin{aligned}g^{m}&\equiv g^{xr} * g^{ks}\\&\equiv (g^{x})^{r} * (g^{k})^{s}\\&\equiv (y)^{r} * (r)^{s}{\pmod {p}}.\\\end{aligned}}}$
\end{itemize}

\section*{Улучшение алгоритма Эль-Гамаля в режиме шифрования и дешифрования для защиты от атаки на основе подобранного шифротекста}
\text{Криптосистема Эль-Гамаля - это криптосистема с открытым ключом, поэтому злоумышленник может использовать атаку подобранного шифротекста на криптосистему Эль-Гамаля без знания закрытого ключа.}
\subsection*{Атака на основе подобранного шифротекста}
    \text{Для начала рассмотрим алгоритм данной атаки.}
\begin{itemize}
    \item Шифротекст, перехваченный злоумышленником - $C = (a, b) = (g^k, M * y^k)$
    \item Злоумышленник выбирает k' и M' случайным образом и вычисляет \\
    $C' = (g^k * g^{k'}, b^k * M * b^{k'} * M') = (g^{k + k'}, (M*M') * b ^ (k + k'))$
    \item Злоумышленник пытается произвести дешифрование с оракулом (алгоритм которого не позволяет дешифровки шифротекста C) шифротекста C', и возвращается M''
    \item Теперь из M'' можно получить M :\\
    $C'' = (M * M') * b ^ {k + k'}$\\
    $M'' = M * M'$\\
    $M = M'' * M'^{-1}$
\end{itemize}

\subsection*{Предложенное улучшение схемы Эль-Гамаля}
\text{Допустим, нам нужно зашифровать сообщение M. Закрытый и открытый ключ аналогичны ключам базового алгоритма. Структура зашифрованного текста C изменилась в улучшенной системе.}
\\
\newline
    \caption{Шифрование:}
    \begin{itemize}
        \item Запишем M как $M = p_1p_2...p_i$, где $p_i$ - простые числа, (0 < i < p).
        \item Вычислим $n = i * y ^ k\pmod{p}$ и $b = M ^ i$
        \item Отправляем (a, b, n)
    \end{itemize}
\\
    \caption{Расшифрование:}
    \begin{itemize}
        \item Вычисляем $a^x \pmod{p} = y ^ k$
        \item Вычислим $i = \frac{n}{a^x \pmod{p}}$
        \item Считаем обратный элемент к i и вычисляем сообщение M: $M = b ^{1/i}$
    \end{itemize}
\\
\caption{Корректность алгоритма:}\\
\text{$b^{\frac{a^x mod p}{n}} = b^{\frac{g^{k * x} mod p}{i * y ^ {k} mod p}} = b^{\frac{g^{k * x} mod p}{i * g ^ {x * k} mod p}} = b ^{1/i} = (b^i) ^ (1/i) = M$}

\subsection*{Устойчивость к выбранной атаке}
\text{Теперь злоумышленник получает зашифрованный текст C = (a, b, n).
Злоумышленник выбирает значения k', M’ и t случайным образом. (Согласно
предыдущей атакой на криптосистему с открытым ключом Эль-Гамаля,
злоумышленник выбирает только два случайных значения. С двумя значенями он
никогда не сможет атаковать эту расширенную систему. Итак, атакующий выбирает
3 значения для атаки на эту расширенную систему).\\
$C = (a, b, n) = (g^k', M^i, i * y^k mod p)$\\
Теперь $C' = (g^k' * g ^ t, M^i * M'^t, y^k mod p * t * y^k' mod p)$\\
$C' = (g^{k + k'}, M^i * M'^k', (i * t) * (y^k mod p) * (y^k' mod p))$\\
Теперь из функции расшифрования возвращается M''.\\
$M'' = M^i * M'^t$\\
$M = (\frac{M''}{M'^t})^{1/i}$\\
Значение i злоумышленнику неизвестно, следовательно, он не может вычислить M из M''.}

\section*{Схема цифровой подписи Эль-Гамаля на эллиптической кривой над кольцом $Z_n$}
\text{Схема Эль-Гамаля основана на задаче дискретного логарифмирования (DLP). Безопасность RSA основана на проблеме факторизации больших целых чисел (IFP). Рассмотрим улучшение схемы цифровой подписи типа Эль-Гамаля, основанное на конической кривой над кольцом Zn, которая превращает исходную схему цифровой подписи, основанную на одной сложной математической задаче, в основанную на двух сложных математических задачах. Следовательно, улучшенная схема более безопасна, чем исходная схема.}
\subsection*{Выбор параметров}
\text{Выберем эллиптическую кривую с уравнением $C_{n}(a, b):y^{2}\equiv ax^{2}-bx({\bmod}\ n)$, \\
    где n = p * q, a и b $\in Z_n$ ,такие, что (a, n) = (b, n) = 1, p и q - два больших простых числа, такие, что $\frac{a}{p} = \frac{a}{p} = -1$ и p + 1 = 2r, q + 1 = 2s. По определению порядок кривой: $N_n = 2rs$
    Начальная точка кривой: $G = (G_x, G_y)$. Случайным образом выбираем d $\in Z_N_n*$. Считаем $Q = dG (mod n) \neq (0, 0)$. H(m) - хэш сообщения m.
    Публичный ключ n, a, b, Q, g. Приватный ключ - d и $N_n$}

\subsection*{Алгоритмы подписи и проверки}\\
\caption{Подпись: }
\begin{itemize}
    \item Случайным образом выбираем k и считаем \\
    $kG({\bmod}\ n)\equiv(x_{1},y_{1}), \gamma\equiv x_{1}({\bmod}\ N_{n})$(если $\gamma = 0$, то выбираем другое k) 
    \item Вычисляем $R\equiv k\gamma G({\bmod}\ n)$
    \item Вычисляем $\delta\equiv(dH({\rm m})-k\gamma^{2})({\bmod}\ N_{n})$
    \item Отправляем $(m, R, \gamma, \delta)$ в качестве подписи сообщения m
\end{itemize}
\\
\caption{Проверка подписи: }
\begin{itemize}
    \item После получения подписи $(m, R, \gamma, \delta)$ получатель вычисляет $U\equiv H(m)Q({\bmod}\ n), X\equiv\delta G({\bmod}\ n), Y\equiv R({\bmod}\ n)$
    \item Если среди этих значений есть (0, 0), то подпись недействительна
    \item Если верно, что $X\oplus Y=U$, то подпись действительна, иначе - недействительна
\end{itemize}\\
\caption{Корректность проверки :}\\
\text{$\eqalignno{& X{\oplus} Y({\bmod} n) \cr & {\equiv}\delta G{\oplus} R({\bmod} n) \cr & {\equiv}(dH({\rm m})-{\gamma})G{\oplus} R({\bmod} n) \cr & {\equiv}}$ \\
${{{{\equiv} (dH({\rm m})-{\gamma})G{\oplus}{\gamma} G({\bmod} n) \cr & {\equiv} dH({\rm m})G({\bmod} n) \cr & {\equiv} H({\rm m})Q({\bmod} n) \cr & {\equiv} U}}}$}

\subsection*{Симуляция атак}
\begin{itemize}
    \item Если злоумышленник хочет подделать подпись сообщения $\delta=(dH({\rm m})-\gamma)({\bmod}\ N_{n})$, он должен получить $N_n$. Предположим, что злоумышленник способен решить проблему целочисленной факторизации, значит он может разложить n на p и q, а следовательно, узнает параметр $N_n$. Однако, так как мы знаем сложность вычисления дискретного логарифма конической кривой, злоумышленник все еще не может получить приватный ключ d, зная публичный ключ $Q\equiv dG({\bmod}\ n)$.
    \item Если злоумышленник умеет вычислять дискретный логарифм конической кривой, предполагаем, что у него есть приватный ключ d из публичного ключа  $Q\equiv dG({\bmod}\ n)$. Чтобы подделать подпись, ему нужно вычислить значение $\delta=(dH({\rm m})-\gamma)({\bmod}\ N_{n})$. Так как ему не известен параметр $N_n$, злоумышленнику нужно разложить n на p и q, то есть решить проблему целочисленной факторизации.
\end{itemize}

\text{Приведенные выше смоделированные атаки показывают, что новая схема цифровой подписи очень безопасна в случае невозможности решения двух сложных математических задач одновременно.}

\section*{Улучшенный алгоритм Эль-Гамаля на основе добавления случайного числа}
\text{С точки зрения схемы цифровой подписи Эль-Гамаля безопасность алгоритма зависит от безопасности закрытого ключа x. При этом хакер может использовать связь между случайными числами, использующимися в алгоритме подписи, чтобы получить значение закрытого ключа x без сложных вычислений.}

\text{Таким образом, в алгоритме цифровой подписи Эль-Гамаля незащищенное случайное число представляет собой очень большую угрозу его безопасности. Из анализа атаки на случайное число, было обнаружено, что хакер может легко вычислить значение ключа путем вычисления случайного числа. Рассмотрим улучшение алгоритма цифровой подписи Эль-Гамаля, устраняющее данную уязвимость путем добавления случайного числа и усложнения связи между случайным числом и закрытым ключом.}

\subsection*{Алгоритмы подписи и проверки подписи}\\
\caption{Подпись :}
\begin{itemize}
    \item Выбирается большое число p, $\alpha$ - генератор ${\rm Zp}^{\ast}, {\rm x}(1 < x < \varphi(p))$ - приватный ключ, соответствующий публичный ключ может быть вычислен как $\beta=\alpha^{x}\bmod\ p$
    \item Выбираются два случайных различных числа t и k, где t и k взаимно простые и существуют обратные к ним числа. Вычисляем $\gamma=\alpha^{k}\bmod\ p, \lambda=\alpha^{t}\bmod\ p$
    \item M, $\delta$ вычисляются с использованием результатов первых двух шагов, а также расширенного алгоритма Евклида и мультипликативной инверсии $M=(x\gamma+k\lambda+t\delta)\bmod(p-1)$, 
    $\delta=(m-k\beta)x^{-1}\bmod(p-1)$
    \item Требуемые открытые ключи: p, $\beta$ и $\alpha$. Закрытый ключ — x. Подпись открытого текста M — $(\gamma, \lambda, \delta)$.
\end{itemize}
\\
\caption{Проверка :}
\begin{itemize}
    \item $(\gamma, \lambda, \delta)$ отправляется системой соответствующим клиентам.
    \item Клиенты используют следующее уравнение для проверки правильности цифровых подписей открытого текста M:\\
    $\alpha^{M}=\beta^{\gamma}\gamma^{\lambda}\lambda^{\delta}\bmod\ p$
    \item Если равенство верно, то подпись считается корректной, в противном случае подпись неверна или была ошибка передачи. 
\end{itemize}
    
\subsection*{Симуляции атак на улучшенный алогоритм}
\text{Для схемы цифровой подписи Эль-Гамаля взлом происходит по двум причинам: необходимость хранения ключа x и возможность подделки подпис без знания ключа x. Исследуем улучшенную схему для некоторых алгоритмов взлома.}
\newline
\begin{itemize}
    \item Если для подписи используется схема цифровой подписи Эль-Гамаля с добавлением случайного числа, то на начальном этапе хакер может определить закрытый ключ, но он не может вычислить два случайных числа k и t из уравнений для $\gamma$ и $\lambda$. 
    Точно так же хакер не сможет использовать $\gamma=\alpha^{k}\bmod\ p, \cr &\lambda=\alpha^{t}\bmod\ p$ для проверки правильности закрытого ключа x, кототорый он узнал. Поэтому в улучшенной схеме цифровой подписи Эль-Гамаля хакер не может подделать подпись, основываясь на формуле $M = (x \gamma+k\lambda+t\delta)\bmod(p-1)$
    \item Таким образом, основная атака будет сосредоточена на уравнении проверки цифровой подписи: $\alpha^{M}=\beta^{\gamma}\gamma^{\lambda}\lambda^{\delta}\bmod\ p$.
    $\delta$ рассчитывается из: $\lambda^{\delta}=\alpha^{m}\beta^{-\gamma}\gamma^{-\lambda}\pmod p$.
    Сравнивая выражения выше, вычисление последнего явно сложнее, так как у последнего на одну инверсию больше, чем у первого.
И даже при успешном взломе сообщения M, оно являются известным значением в процессе расчета, поэтому результаты расчета могут быть использованы только для подписи этого сообщения. Ценность этих атак явно
снижена.
\item Для улучшенных алгоритмов, по сравнению со многими традиционными алгоритмами цифровой подписи, хакер все еще может легко получить доступ к подписи $(\gamma,\delta)$ сообщения M, а затем подделать ряд законных цифровых подписей.

Поскольку для улучшенного алгоритма введено больше параметров, вероятность того, что мы захотим подделать подпись сообщения M’ уменьшена. Даже если способ подделки подписи будет найден, атака на него будет более сложной, чем на алгоритм цифровой подписи Эль-Гамаля. Действительная подпись, полученная таким образом, по-прежнему соответствует значению M.

Методы атаки с добавлением подписанных данных сложнее построить. Такое вычисление намного больше, даже если оно удастся. Но настоящие трудности для хакера по-прежнему представляют собой действующую подпись, которая была подделана в то же время, а также производила соответствующее значение M'. Злоумышленник может успешно найти соответствующее значение M', но хакер, атакующий таким образом, должен иметь очень большую вычислительную способность.
\end{itemize}

\section*{Вывод}
\text{Таким образом, были рассмотрены возможные атаки на алгоритм подписи и шифрования Эль-Гамаля и улучшения, повышающие защищенность системы от выбранных атак. Рассмотренные алгоритмы с точки зрения безопасности были значительно улучшены, что делает их область применения достаточно широкой.
}
\section*{Источники}
\begin{itemize}
    \item Статья "Схема Эль-Гамаля" на https://ru.wikipedia.org/wiki/
    \item Dissanayake W. An Improvement of the Basic El-Gamal Public Key Cryptosystem //International Journal of Computer Applications Technology and Research. – 2018. – Т. 7. – №. 02. – С. 40-44.
    \item Bai C. X. et al. A new digital signature scheme of ElGamal type on conic curve over the ring Z n //2010 International Conference on Computer Application and System Modeling (ICCASM 2010). – IEEE, 2010. – Т. 11. – С. V11-378-V11-381.
    \item Xiao-fei L., Xuan-jing S., Hai-peng C. An improved ElGamal digital signature algorithm based on adding a random number //2010 Second International Conference on Networks Security, Wireless Communications and Trusted Computing. – IEEE, 2010. – Т. 2. – С. 236-240.
\end{itemize}

\end{document}
